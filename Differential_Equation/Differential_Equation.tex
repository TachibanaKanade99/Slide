\documentclass[10pt]{article}
\usepackage{pictex,amsmath,amssymb,amsbsy,amsfonts,amsthm,verbatim}
\usepackage{graphics}
\usepackage{fullpage}
\usepackage{fancyhdr}
\usepackage{algorithm,algorithmic}
\usepackage{multirow}
\usepackage{graphicx}
\setlength{\voffset}{-0.25in}
\setlength{\headsep}{+0.5in}
\setlength{\parskip}{1em}
\setlength{\parindent}{0em}

\def\vu{\mathbf{u}}
\def\vv{\mathbf{v}}
\def\vs{\mathbf{s}}
\def\vb{\mathbf{b}}
\def\vw{\mathbf{w}}

\renewcommand{\implies}{\rightarrow}
\renewcommand{\lor}{\vee}
\renewcommand{\land}{wedge}
\renewcommand{\iff}{\leffrightarrow}
\newcommand{\xor}{\oplus}
\newcommand{\TRUE}{\mathbf{T}}
\newcommand{\FALSE}{\mathbf{F}}
\newcommand{\universe}{\mathcal{U}}

\usepackage{xcolor}
\usepackage{titlesec}
\usepackage{mdframed}
\usepackage[utf8]{vietnam}

\newmdenv[linecolor=blue,skipabove=\topsep, skipbelow=\topsep,leftmargin=5pt,rightmargin=-5pt,innerleftmargin=5pt,innerrightmargin=5pt]{mybox}

\begin{document}
\begin{center}
\textbf{ORDINARY DIFFERENTIAL EQUATIONS}
\end{center}
\begin{enumerate}
	%1
	\item \textbf{Separable Equations}\\
	- A \textbf{separable equation} is a first-order differential equation, which has a form:
	\begin{mybox}
	\begin{center}
	$P(x)dx + Q(y)dy = 0$
	\end{center}
	\end{mybox}
	\textbf{\underline{Solution:}}\\
	Integrate both sides, we receive:
	\begin{center}
	$\displaystyle \int P(x)dx +\int Q(y)dy = C$
	\end{center}
	%2
	\item \textbf{The differential Equation of the form}
	\begin{mybox}
	\begin{center}
	$f_1(x).g_1(y)dx + f_2(x).g_2(y)dy = 0$
	\end{center}
	\end{mybox}
	\textbf{\underline{Solution:}}\\
	\begin{equation}
	\mbox{If } f_1(x),f_2(x),g_1(y),g_2(y) \not = 0, \mbox{then we divide both sides by } f_2(x).g_1(y). \mbox{ So:}
	\end{equation}
	\begin{center}
	$\dfrac{f_1(x)}{f_2(x)}dx + \dfrac{g_2(x)}{g_1(x)} = 0$
	$\rightarrow{\displaystyle \int \dfrac{f_1(x)}{f_2(x)}dx + \int \dfrac{g_2(y)}{g_1(y)}dy}$
	\end{center}
	%3
	\item \textbf{COOL PROBLEM}\\
	(Liên quan đến the rate of change cần chú ý)\\
	- The rate of  change of the temperature T(t) is the derivative of $\dfrac{dT}{dt}$\\
	- Acording to the Newton's law of cooling:
	\begin{mybox}
	\begin{center}
	$\dfrac{dT}{dt} = k(T-T_s)$
	\end{center}
	\end{mybox}
	When k is a constant of propotionality.\\
	Separating the variables we have:
	\begin{center}
	\begin{align}
	\dfrac{dT}{T-T_s} = kdt \leftrightarrow \dfrac{dT}{T-T_s} = kdt \\
	\rightarrow \displaystyle \int \dfrac{dT}{T-T_s} = k \int dt \rightarrow ln|T-T_s| = kt + lnC \\
	\rightarrow T -T_s = e^{kt + lnC} = e^{kt}.e^{lnC} = Ce^{kt}
	\end{align}
	\end{center}
	%4
	\item \textbf{The first-order linear differential equation}\\
	- The first-order \textbf{linear} differential equation is one that can be put into the form:
	\begin{mybox}
	\begin{center}
	\begin{equation}
	\dfrac{dy}{dx} + P(x).y = Q(x)
	\end{equation}
	\end{center}
	\end{mybox}
	- To solve the linear diferential equation, multply both side by the \textbf{integrate factor} $e^{P(x)dx}$ and integrate both sides:
	\begin{mybox}
	\begin{center}
	\begin{equation}
	y = e^{- \int P(x)dx}.{\displaystyle \int Q(x).e^{\int P(x)dx}dx + C}
	\end{equation}
	\end{center}
	\end{mybox}
\end{enumerate}
\pagebreak
\begin{center}
\textbf{SECOND ORDER LINEAR DIIFERENTIAL EQUATIONS WITH CONSTANT COEFFICIENTS}
\end{center}
\begin{enumerate}
	%1
	\item \textbf{Homogeneous and Non-Homogeneous Second Differential Equations}.\\
	- Homogeneous Second Differential Equation has a form:
	\begin{mybox}
	\begin{center}
	\begin{equation}
	Ay" + By' + Cy = 0 (A,B,C \in \mathbb{R}; A \not = 0)
	\end{equation}
	\end{center}
	\end{mybox}
	- Non-homogeneous Second Differential Equation has a form:
	\begin{mybox}
	\begin{center}
	\begin{equation}
	Ay" + By' + Cy = f(x) (A,B,C \in \mathbb{R}; A \not = 0)
	\end{equation}
	\end{center}
	\end{mybox}
	%2
	\item Let $\textbf{$y_1,y_2$}$ be any 2 solutions of the homogeneous linear differential equation (1), then:
	\begin{center}
	$\textbf{$C_1y_1 + C_2y_2$}$
	\end{center}
	is also a \textbf{solution} of equation (1) where $C_1, C_2$ are arbitrary constants.(hàm tùy ý)
	\begin{itemize}
	\item The 2 function $y_1,y_2$ are \textbf{linear dependent} on $a \le x \le b$ if \textit{there exists constants $C_1,C_2$ not zero}, such that:
	\begin{center}
	$C_1y_1 + C_2y_2 = 0$
	\end{center}
	\item The 2 function $y_1,y_2$ are \textbf{linear independent} on $a \le x \le b$ if:
	\begin{center}
	$C_1y_1 + C_2y_2 = 0$
	\end{center}
	For all x such that $a \le x \le b$ implies $C_1 = C_2 = 0$.
	\end{itemize}
	%3
	\item Let $y_1,y_2$ be 2 real function each of which has the first derivative on the interval $a \le x \le b$. The determinant (định thức):
	\begin{center}
	\begin{align*}
	W(x) = 
	\begin{vmatrix}
	y_1(x) & y_2(x)\\
	y'_1(x) & y'_2(x)
	\end{vmatrix}
	= y_1(x) \times y'_2(x) - y'_1(x) \times y_2(x)
	\end{align*}
	\end{center}
	%4
	\item If $y_1,y_2$ are 2 linearly independent solutions of equation (7), then every \textbf{homogeneous solution} \textit{$y_h$} is:
	\begin{center}
	$y_h = C_1y_1 + C_2y_2$
	\end{center}
	We have equation (7) as:
	\begin{center}
	$Ay" + By' + Cy = 0$
	\end{center}
	Consider that:
	\begin{align*}
	y = e^{kx}\\
	y' = ke^{kx}\\
	y" = k^2e^{kx}
	\end{align*}
	We receive:
	\begin{align}
	Ak^2e^{kx} + Bke^{kx} + Ce^{kx} = 0\\
	\leftrightarrow{Ak^2 + Bk + C = 0}
	\end{align}
	From equation(10), we have evaluate $\Delta$ of the equation:
	\begin{itemize}
		\item $\Delta > 0$, the equation has \textbf{real and different roots $k_1,k_2 \not = 0$}.Then:
		\begin{center}
		$y_1(x) = e^{k_1x}$\\
		$y_2(x) = e^{k_2x}$
		\end{center}
		The homogeneous solution is:
		\begin{mybox}
		\begin{center}
		 $y_h = C_1e^{k_1x} + C_2e^{k_2x}$
		 \end{center}
		 \end{mybox}
		 \item $\Delta = 0$ then the characteristic equation has \textbf{double real root}: $k = k_{0}$\\
		 The the homogeneous solution is:
		 \begin{mybox}
		 \begin{center}
		 $y_h = C_1e^{k_0x} + C_2xe^{k_0x}$
		 \end{center}
		 \end{mybox}
		 \item $\Delta < 0$, then characteristic equation has \textbf{complex conjugate roots $k_1 = a + bi$ and $k_2 = a - bi$}\\
		 Then the homogeneous solution is:
		 \begin{mybox}
		 \begin{center}
		 $y_h = e^{ax} (C_1cos(bx) + C_2sin(bx))$
		 \end{center}
		 \end{mybox} 
	\end{itemize}
	%5
	\item \textbf{Non-homogeneous Equation with constant coefficients}
	\begin{itemize}
		\item \textbf{Case I:$f(x) = e^{\alpha x}.P_n(x)$}\\
		The particular solution of non-homogeneous equation has a form $y_p = x^s.e^{\alpha x}.Q_n(x)$
		\begin{itemize}
			\item If $\alpha$ is not the root of the characteristic equation then s =0 and the particular equation has a form $y_p = e^{\alpha x}.Q_n(x)$
			\item If $\alpha$ is one the 2 different roots of the characteristic equation then s =1 and the particular solution has a form $y_p x.e^{\alpha x}.Q_n(x)$
			\item If $\alpha$ is the double root of the characteristic equation then s = 2 and the particular equation has a form $y_p = x^2.e^{\alpha x}.Q_n(x)$
		\end{itemize}
		\item \textbf{Case 2}: f(x) = $e^{\alpha x}.[P_n(x).cos(\beta x).Q_m(x).sin(\beta x)]$\\
		The particular solution of non-homogeneous equation has a form:
		\begin{center}
		$y_p = x^s.e^{\alpha x}.[H_k(x).cos(\beta x) + T_m(x).sin(\beta x)]$
		\end{center}
		\begin{itemize}
			\item If $\alpha + i \beta$ \textbf{is not the root} of the characteristic equation then \textbf{s = 0} and the particular solution has a form $y_p = e^{\alpha x}.[H_k(x).cos(\beta x) + T_k(x).sin(\beta x)]$
			\item If $\alpha + i \beta$ \textbf{is one of the complex conjugate roots} of the characteristic equation then \textbf{s = 1} and the particular solution has a form $y_p = x.e^{\alpha x}.[H_k(x).cos(\beta x) + T_k(x).sin(\beta x)]$
		\end{itemize}
	\end{itemize}	
		%6
		\item \textbf{Find the non-homogeneous equation:}\\
		- Step 1:Solve the homogeneous equation, use characteristic equation to find $k_{1},k_{2}$.\\
		- Step 2:Evaluate the homogeneous equation.\\
		- Step 3:Find a particular solution of non-homogeneous equation. 
\end{enumerate}		   
\end{document}	   