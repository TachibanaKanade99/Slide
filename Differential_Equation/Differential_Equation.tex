\documentclass[10pt]{article}
\usepackage{pictex,amsmath,amssymb,amsbsy,amsfonts,amsthm,verbatim}
\usepackage{graphics}
\usepackage{fullpage}
\usepackage{fancyhdr}
\usepackage{algorithm,algorithmic}
\usepackage{multirow}
\usepackage{graphicx}
\setlength{\voffset}{-0.25in}
\setlength{\headsep}{+0.5in}
\setlength{\parskip}{1em}
\setlength{\parindent}{0em}

\def\vu{\mathbf{u}}
\def\vv{\mathbf{v}}
\def\vs{\mathbf{s}}
\def\vb{\mathbf{b}}
\def\vw{\mathbf{w}}

\renewcommand{\implies}{\rightarrow}
\renewcommand{\lor}{\vee}
\renewcommand{\land}{wedge}
\renewcommand{\iff}{\leffrightarrow}
\newcommand{\xor}{\oplus}
\newcommand{\TRUE}{\mathbf{T}}
\newcommand{\FALSE}{\mathbf{F}}
\newcommand{\universe}{\mathcal{U}}

\usepackage{xcolor}
\usepackage{titlesec}
\usepackage{mdframed}
\usepackage[utf8]{vietnam}

\newmdenv[linecolor=blue,skipabove=\topsep, skipbelow=\topsep,leftmargin=5pt,rightmargin=-5pt,innerleftmargin=5pt,innerrightmargin=5pt]{mybox}

\begin{document}
\begin{center}
\textbf{ORDINARY DIFFERENTIAL EQUATIONS}
\end{center}
\begin{enumerate}
	%1
	\item \textbf{Separable Equations}\\
	- A \textbf{separable equation} is a first-order differential equation, which has a form:
	\begin{mybox}
	\begin{center}
	$P(x)dx + Q(y)dy = 0$
	\end{center}
	\end{mybox}
	\textbf{\underline{Solution:}}\\
	Integrate both sides, we receive:
	\begin{center}
	$\displaystyle \int P(x)dx +\int Q(y)dy = C$
	\end{center}
	%2
	\item \textbf{The differential Equation of the form}
	\begin{mybox}
	\begin{center}
	$f_1(x).g_1(y)dx + f_2(x).g_2(y)dy = 0$
	\end{center}
	\end{mybox}
	\textbf{\underline{Solution:}}\\
	\begin{equation}
	\mbox{If } f_1(x),f_2(x),g_1(y),g_2(y) \not = 0, \mbox{then we divide both sides by } f_2(x).g_1(y). \mbox{ So:}
	\end{equation}
	\begin{center}
	$\dfrac{f_1(x)}{f_2(x)}dx + \dfrac{g_2(x)}{g_1(x)} = 0$
	$\rightarrow{\displaystyle \int \dfrac{f_1(x)}{f_2(x)}dx + \int \dfrac{g_2(y)}{g_1(y)}dy}$
	\end{center}
	%3
	\item \textbf{COOL PROBLEM}\\
	(Liên quan đến the rate of change cần chú ý)\\
	- The rate of  change of the temperature T(t) is the derivative of $\dfrac{dT}{dt}$\\
	- Acording to the Newton's law of cooling:
	\begin{mybox}
	\begin{center}
	$\dfrac{dT}{dt} = k(T-T_s)$
	\end{center}
	\end{mybox}
	When k is a constant of propotionality.\\
	Separating the variables we have:
	\begin{center}
	\begin{align}
	\dfrac{dT}{T-T_s} = kdt \leftrightarrow \dfrac{dT}{T-T_s} = kdt \\
	\rightarrow \displaystyle \int \dfrac{dT}{T-T_s} = k \int dt \rightarrow ln|T-T_s| = kt + lnC \\
	\rightarrow T -T_s = e^{kt + lnC} = e^{kt}.e^{lnC} = Ce^{kt}
	\end{align}
	\end{center}
	%4
	\item \textbf{The first-order linear differential equation}\\
	- The first-order \textbf{linear} differential equation is one that can be put into the form:
	\begin{mybox}
	\begin{center}
	\begin{equation}
	\dfrac{dy}{dx} + P(x).y = Q(x)
	\end{equation}
	\end{center}
	\end{mybox}
	- To solve the linear difernetial equation, multply both side by the \textbf{integrate factor} $e^{P(x)dx}$ and integrate both sides:
	\begin{mybox}
	\begin{center}
	\begin{equation}
	y = e^{- \int P(x)dx}.{\displaystyle \int Q(x).e^{\int P(x)dx}dx + C}
	\end{equation}
	\end{center}
	\end{mybox}
\end{enumerate}
\end{document}	   