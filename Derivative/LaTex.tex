\documentclass[10pt]{article}
\usepackage{pictex,amsmath,amssymb,amsbsy,amsfonts,amsthm,verbatim}
\usepackage{graphics}
\usepackage{fullpage}
\usepackage{fancyhdr}
\usepackage{algorithm,algorithmic}
\usepackage{multirow}
\setlength{\voffset}{-0.25in}
\setlength{\headsep}{+0.5in}
\setlength{\parskip}{1em}
\setlength{\parindent}{0em}

\newcounter{problem}
\newcommand{\problem}{\textbf{\refstepcounter{problem}Problem \theproblem}}

\def\vu{\mathbf{u}}
\def\vs{\mathbf{x}}
\def\vb{\mathbf{b}}
\def\vv{\mathbf{v}}
\def\vw{\mathbf{w}}

\renewcommand{\implies}{\rightarrow}
\renewcommand{\lor}{\vee}
\renewcommand{\land}{wedge}
\renewcommand{\iff}{\leffrightarrow}
\newcommand{\xor}{\oplus}
\newcommand{\TRUE}{\mathbf{T}}
\newcommand{\FALSE}{\mathbf{F}}
\newcommand{\universe}{\mathcal{U}}

\usepackage{xcolor}
\usepackage{titlesec}
\usepackage{mdframed}
\usepackage{amsmath}
\usepackage[utf8]{vietnam}
 
\newmdenv[linecolor=black,skipabove=\topsep,skipbelow=\topsep,
leftmargin=-5pt,rightmargin=-5pt,
innerleftmargin=5pt,innerrightmargin=5pt]{mybox}

\begin{document}
\textit{\textbf{SOME CALCULUS FUNCTIONS }}\\
 \textbf{DERIVATIVES OF UNIVERSE TRIGONOMETRIC FUNTIONS}
\begin{itemize}
	\item $y = arcsinx (x \in (-1;1))$
	$\rightarrow{y' = \dfrac{1}{\sqrt{1-x^2}}}$
	\item $y = arccosx (x \in (-1;1) )$
	$\rightarrow{y' = \dfrac{-1}{\sqrt{1-x^2}}}$
	\item $y = arctanx$
	$\rightarrow{y' = \dfrac{1}{1+x^2}}$
	\item y = arccotx
	$\rightarrow{y' = \dfrac{-1}{1+x^2}}$
\end{itemize}
 \textbf{DERIVATIVE OF HYPERBOLIC FUNCTION}
\begin{itemize}
    \item $y=sinhx \rightarrow{y'=coshx}$
    \item $y=coshx \rightarrow{y'=sinhx}$
    \item $y=tanhx \rightarrow{y'=\dfrac{1}{cosh^{2}x}}$
    \item $y=cothx \rightarrow{y'=\dfrac{-1}{sinh^{2}x}}$
\end{itemize}
 \textbf{{LEINIZ FORMULA}}\\
"If f(x) and g(x) have $n^{th} \mbox{ derivative then }f(x) \times g(x) \mbox{ also have }n^{th}$ derivative"\\
\begin{center}
$(f(x)\times g(x))^{n} = \displaystyle \sum_{k=0}^{n}(C^k_nf^{n-k}(x)g^k(x))$\\
\end{center}
 \textbf{SOME BASIC FORMULA}\\
\begin{itemize}
	\item $(a^x)^n = a^{x}ln^{n}a$
	\item $(e^x)^n = e^{x}$
	\item $(sinax)^n = a^nsin(ax + \dfrac{n\pi}{2})$
	\item $(cosax)^n = a^ncos(ax + \dfrac{n\pi}{2})$
	\item $((ax+b)^{\alpha})^n = a^n\alpha(\alpha-1)\ldots(\alpha-n+1)(ax+b)^{\alpha-n}$
	\item $log_a{|x|}^{n} = \dfrac{(-1)^{n-1}(n-1)!}{x^{n}lna}$
	\item $ln|x|^n = \dfrac{(-1)^{n-1}(n-1)!}{x^n}$
\end{itemize}
 \textbf{LINEAR APPROXIMATION}\\
$f(x) \approx f(a) + f'(a)(x-a)$ is called the linear approximation or tangent line approximation of f at a\\
\pagebreak
 \textbf{THE 1ST ORDER DIFFERENTIAL}\\
"The 1st order differential dy  of y = f(x) at a is defined in term of dx by equation''
\begin{mybox}
\begin{center}
df(a)= f'(a)dx
\end{center}
\end{mybox}
 \textbf{THE 2ND ORDER DIFFERENTIAL}\\
``The 2nd order differential of y = f(x) at a is defined in term of dx by equation''
\begin{mybox}
\begin{center}
$d^2f(a)=f"(a)dx^2$
\end{center}
\end{mybox}
 \textbf{THE NTH ORDER DIFFERENTIAL}\\
``The nth order differential of y = f(x) at a is defined in term of dx by equation''
\begin{mybox}
\begin{center}
$d^nf(a)=f^{(n)}(a)dx^n$
\end{center}
\end{mybox}
\end{document}	
