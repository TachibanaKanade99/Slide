\documentclass[10pt]{article}
\usepackage{pictex,amsmath,amssymb,amsfonts,amsthm,verbatim}
\usepackage{graphics,graphicx}
\usepackage{fullpage}
\usepackage{multirow}
\usepackage{mathrsfs}
\usepackage[version=4]{mhchem}
\usepackage{xcolor}
\usepackage{titlesec}
\usepackage{mdframed}
\usepackage[utf8]{vietnam}

\newmdenv[linecolor=red,skipabove=\topsep.skipbelow=\topsep,leftmargin=5pt,rightmargin=-5pt,innerleftmargin=5pt,innerrightmargin=5pt]{mybox}

\setlength{\voffset}{-0.25in}
\setlength{\headsep}{+0.5in}
\setlength{\parskip}{1em}
\setlength{\parindent}{0em}

\def\vu{\mathbf{u}}
\def\vs{\mathbf{s}}
\def\vv{\mathbf{v}}
\def\vw{\mathbf{w}}
\def\vb{\mathbf{b}}

\renewcommand{\implies}{\rightarrow}
\renewcommand{\lor}{\vee}
\renewcommand{\land}{wedge}
\renewcommand{\iff}{\leffrightarrow}
\newcommand{\TRUE}{\mathbf{T}}
\newcommand{\FALSE}{\mathbf{F}}
\newcommand{\universe}{\mathcal{U}}

\begin{document}
\begin{center}
	\textbf{THERMOCHEMISTRY}
\end{center}
\begin{enumerate}
	%1
	\item \textbf{Enthalpy of Chemical Reaction}\\
	\begin{mybox}
	\begin{center}
	$H = E + PV$
	\end{center}
	\end{mybox}
	The change in Enthalpy:
	\begin{mybox}
	\begin{center}
	$\Delta H = \Delta E + \Delta (PV)$
	\end{center}
	\end{mybox}
	If the pressure is held constant:
	\begin{center}
	$\Delta H = \Delta E + P \Delta V$
	\end{center}
	\textbf{Enthalpy of Reaction}\\
	- Because most reactions are constant-pressure process, we can equate the heat change in these cases to the change in enthalpy.
	\begin{center}
	\ce{ reaction -> products}
	\end{center}
	$\rightarrow{\mbox{The change in enthalpy, called the \textbf{Enthalpy of Reaction,$\Delta H$}}}$.\\
	\begin{mybox}
	\begin{center}
	$\Delta H = H(products) - H(reactants)$
	\end{center}
	\end{mybox}
	\begin{itemize}
		\item $\Delta H >0$, the reaction is an endothermic process.
		\item $\Delta H < 0$, the reaction is an exorthermic process.
	\end{itemize}	
\end{enumerate}
\end{document}	   		